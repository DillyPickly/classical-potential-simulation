\documentclass{article}
\setlength{\parindent}{0em}
\setlength{\parskip}{0.5em}
\begin{document}


\section*{A Simple Python Simulator}
    In classical physics, the EOM of a system can be determined from the potential in the system.
    Here, I will investigate a discrete particle simulation
    The project consists of three main python files.
    \begin{itemize}
        \item \textbf{config.py}: holds all the configuration information for the project.
        \item \textbf{driver.py}: computes the particle dynamics and saves them to a file.
        \item \textbf{visualize.py}: creates animations to view the simulations.
    \end{itemize}  

\vspace{5pt}
\hrule
\vspace{5pt}

\subsection*{Physics Relationships}

Looking at Potential Energy, we have:

\begin{equation}
       \overrightarrow{\mathbf{F}} = -\nabla \mathbf{U} = \langle -\frac{\partial \mathbf{U}}{\partial x}, -\frac{\partial \mathbf{U}}{\partial y} \rangle
\end{equation}

\begin{equation}
    \overrightarrow{\mathbf{F}} = m\overrightarrow{\mathbf{a}}
\end{equation}

\noindent The time derivative of velocity is acceleration so assuming that the potential energy is time independant.
The change in velocity and position from $t_n \rightarrow t_{n+1}$ is:

\begin{equation}
    \Delta \overrightarrow{\mathbf{v}} = \overrightarrow{\mathbf{a}} (t_{n+1} - t_{n})
\end{equation}

\begin{equation}
    \Delta \overrightarrow{\mathbf{x}} = \overrightarrow{\mathbf{v}} (t_{n+1} - t_{n})
\end{equation}

\subsection*{Inelastic vs. Elastic Collisions}

In both types of collisions, the total momentum before the collision is equal to the total colision after the collision.

Perfectly elastic collisions also conserve kinetic energy. This constraint results makes the objects bounce off one another.

On the other hand, perfectly Inelastic collisions do not conserve kinetic energy. Instead a perfectly Inelastic collisions 


Real collisions are somewhere in between perfectly elastic ans perfectly inelastic.
These collisions can be modelled with a coeffiecient of restitution (COR).

For a collision, we have:

\begin{equation}
    m_{1}\overrightarrow{v_{1b}} + m_{2}\overrightarrow{v_{2b}} = m_{1}\overrightarrow{v_{1a}} + m_{2}\overrightarrow{v_{2a}}
\end{equation}

\begin{equation}
    C_{R} =  -\frac{\overrightarrow{v_{1a}} - \overrightarrow{v_{2a}}}{\overrightarrow{v_{1b}} - \overrightarrow{v_{2b}}}
\end{equation}

The coeffiecient of restitution then is defined as the amount of relative velocity that the system keeps after the collision.
In one dimension using the $C_R$ and the conservation of momentum, we obtain:
\begin{equation}
    \overrightarrow{v_{1a}} =  \frac{m_1\overrightarrow{v_{1b}} + m_2\overrightarrow{v_{2b}} + m_1C_R(\overrightarrow{v_{2b}} - \overrightarrow{v_{1b}})}{m_1 + m_2}
\end{equation}

\begin{equation}
    \overrightarrow{v_{2a}} =  \frac{m_1\overrightarrow{v_{1b}} + m_2\overrightarrow{v_{2b}} + m_2C_R(\overrightarrow{v_{1b}} - \overrightarrow{v_{2b}})}{m_1 + m_2}
\end{equation}

\vspace{5pt}
\hrule
\vspace{5pt}

Since the above equations are in one dimendion, we need to project the velocities into coordinates normal and parallel to the collision.

Given a velocity $\overrightarrow{v} = v_x \hat{\textit{\i}} + v_y \hat{\textit{\j}}$ with a collision vector, $v_c$, and normal vector, $v_n$.

\begin{equation}
    \overrightarrow{v_c} = (\overrightarrow{v} \cdot \widehat{v}_c) \widehat{v}_c
\end{equation}

\begin{equation}
    \overrightarrow{v_n} = (\overrightarrow{v} \cdot \widehat{v}_n) \widehat{v}_n
\end{equation}

So we can get the collision components and treat the colision as one dimensional. 
Once I find the change in velocity from the collision, I can resemble the vector in terms of $\hat{\textit{\i}}$ and $\hat{\textit{\j}}$.



\end{document}

