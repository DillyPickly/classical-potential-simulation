\documentclass{article}
\begin{document}


\section*{A Simple Python Simulator}
    In classical physics, the EOM of a system can be determined from the potential in the system.
    Here, I will investigate a discrete particle simulation
    The project consists of three main python files.
    \begin{itemize}
        \item \textbf{config.py}: holds all the configuration information for the project.
        \item \textbf{driver.py}: computes the particle dynamics and saves them to a file.
        \item \textbf{visualize.py}: creates animations to view the simulations.
    \end{itemize}  

\vspace{5pt}
\hrule
\vspace{5pt}

\subsection*{Physics Relationships}

Looking at Potential Energy, we have:

\begin{equation}
       \overrightarrow{\mathbf{F}} = -\nabla \mathbf{U} = \langle -\frac{\partial \mathbf{U}}{\partial x}, -\frac{\partial \mathbf{U}}{\partial y} \rangle
\end{equation}

\begin{equation}
    \overrightarrow{\mathbf{F}} = m\overrightarrow{\mathbf{a}}
\end{equation}

\noindent The time derivative of velocity is acceleration so assuming that the potential energy is time independant.
The change in velocity and position from $t_n \rightarrow t_{n+1}$ is:

\begin{equation}
    \Delta \overrightarrow{\mathbf{v}} = \overrightarrow{\mathbf{a}} (t_{n+1} - t_{n})
\end{equation}

\begin{equation}
    \Delta \overrightarrow{\mathbf{x}} = \overrightarrow{\mathbf{v}} (t_{n+1} - t_{n})
\end{equation}


\end{document}

